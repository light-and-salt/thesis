%考核
\chapter*{\centerline{\stfangsong\xiaoer 毕~业~论~文~(设计)~考~核}}
\thispagestyle{empty}

\vspace{2em}

{\stfangsong\sihao\bf
一、\;指导教师对毕业论文(设计)的评语:

本文阐述了数据命名网络上游戏软件的架构选择,数据命名方案和同步机制,探讨了游戏同步机制与数据命名网络相结合后给游戏软件性能带来的提高。上述理论经验已成功用于样板游戏的开发,生成了一个运行于数据命名网络上的P2P游戏软件实例。所提出的同步机制以函数库的形式发布,并在实际开发过程中封装成为游戏引擎插件。本项目中发布的库函数和插件可以被后续开发人员重用。

论文条理清楚,论述完整,引文确切,达到了本科毕业论文的要求,同意提交答辩。

\begin{flushright}
    指导教师(签名)\;\underline{\hspace{4em}}\\
    年\quad 月\quad 日
\end{flushright}
\vspace{6cm}

二、\;答辩小组对毕业论文(设计)的答辩评语及总评成绩:
\vspace{4cm}

{\songti\xiaosi
\begin{tabular}{|c|c|c|c|c|c|}
    \hline
    成绩比例 & \parbox[t]{4em}{文献综述\\[-3.5em]占(10\%)} & 
               \parbox[t]{4em}{开题报告\\[-3.5em]占(20\%)} & 
               \parbox[t]{4em}{外文翻译\\[-3.5em]占(10\%)} &
               \parbox[t]{7em}{毕业论文(设计)\\[-3.5em]质量及答辩\\[-3.5em]占(60\%)} & 
               总评成绩 \\
    \hline
    分值 & & & & & \\
    \hline
\end{tabular}
}

\begin{flushright}
    答辩小组负责人(签名)\;\underline{\hspace{4em}}\\
    年 \quad 月 \quad 日
\end{flushright}
}
